\documentclass[a4paper]{article}
\usepackage[utf8]{inputenc}
\usepackage[T1]{fontenc}
\usepackage{authblk}
\usepackage{apacite}
\usepackage{natbib}
\usepackage{listings}
\usepackage{graphicx}
\usepackage{hyperref}

\title{Improvements over LVox: an algorithm for estimating forest
  plot's biomass using point clouds}

\author[1]{Félix Chabot}
\author[2]{Richard Fournier}
\author[1]{Toby Dylan Hocking}
\author[2]{Camille Rouet}
\author[2]{Amélie Juckler}
\author[2]{Johannie Lemelin}
\affil[1]{Département de sciences informatiques, Université de Sherbrooke}
\affil[2]{Département de géomatique appliquée, Université de Sherbrooke}

\begin{document}

\maketitle{}

\tableofcontents{}

\section{Introduction}
Laser scanning technologies have been in the forefront of many fields
when it comes to acquiring real life volumetric data. Indeed, not many
other technologies can provide the same amount of precision and
accuracy in a short amount of time. This is why they can be relied
upon in time critical tasks such as autonomous driving and in
situations where you need millimeter level of measurements, in civil
engineering for instance.

Another use case that can benefit greatly from this technology is
forestry. Forestry researchers can be interested to seek a better
understanding of the distribution of forest element in space. To do
so, they often resort to physical measurements. These tasks can be
time-consuming, error-prone and in some cases even destructive. For
example, in order to get a proper estimation of biomass, operators can
be required to cut down a tree to then weight it afterwards. Given the
fact that forest researchers are often interested in large forested
areas, having an easier way to accurately quantify the forest
structure in its entirety would greatly simplify this task. Doing this
estimation in three dimensions with lidar technology is precisely the
aim of the L-Vox (\cite{nguyen2022}) software.

L-Vox, when using multiple scans from a terrestrial laser scanner
(TLS), yields accurate estimation of the Plant Area Density (PAD), an
indicator of the amount of vegetation surface inside a volume. Since
it estimates for a discrete volume in space, This is known as an area
based approach, in opposition to the object based approach which seeks
to look at the individual object of interest (a tree for example) to
estimate a value. This is advantageous when we seek to have a better
understanding of the structure of the forest plot as whole.

The mathematical framework used by L-Vox was first introduced by
\cite{pimont2019}. It is based on a stochastic approach where the
points in the point clouds are seen as statistical events. Two
categories of estimators are defined, the ones that use the inversion
of Beer-Lambert's law or the contact frequency, all of them are
available inside L-Vox. The interaction between the amount of observed
returns inside a given volume and how much it was explored by laser
beams is what is eventually used to estimate the PAD. Since we seek to
estimate for a large area of interest, we need to discretize the area
itself into multiple smaller volumes. This is why the estimation uses
a three-dimensional grid of voxels.

L-Vox was developped in \texttt{C++} as a plugin inside the
\textit{Computree} computing platform (\cite{computree}). This
application platform can be used to do various operations on point
clouds, which are mainly aimed at forestry. It is an open-source
software built with the \texttt{QT} (\cite{QT}) framework to offer a
graphical user interface. The interface contains a built-in
visualization window which can be used to display the content of the
point cloud and the various transformation steps that can be applied
on it.

Though \textit{Computree} offers a myriad of advantages for the
researchers who uses it, from an operational standpoint it isn't
ideal. Firstly, the graphical interface, while useful for developping
new process chains for point cloud data, makes reusing an extisting
one cumbersome. The user is expected to either change the values
directly in the various modal windows before starting the processing,
generate a new script per parameter that needs to be changed or with
all the duplicated parameters of interest with their respective fixed
values in a single script. All these operations require a lot of
manual actions inside the graphical interface. \texttt{Computree} does
offer a <<batch>> mode that doesn't need to be operated from with a
graphical interface, but it requires a script that was generated with
the interface beforehand.

A common use case for operational research, or even for forest
inventories, is to run the same computation on multiple files. Doing
manual operations in the interface for every files be would be
impractial, especially since it would not be unsual for operators to
analyze close to a hundred files. This why the industry has shown a
great interest for software solution that offer some sort of
pipeline. For instance, one of the more widely use tool for processing
point cloud data is the \texttt{R} package \texttt{lidR}
(\cite{roussel2020}). Instead of having a graphical interface, this
package provides a programming interface. Users are expected to write
\texttt{R} code that in turn will compose their processing
pipeline. For handling large amount of files, users can simply define
the processing for one file and leverage the capabilities of the
\texttt{R} language to reuse it for a collection of files. One benefit
that \texttt{Computree} has over \texttt{lidR} is that it doesn't
require the user to have prior programming knowledge. Although, it can
be argued that all the benefits gained from the features of the
programming language (reproducibility, interoperability,
documentation, ect.) outweights the cost of entry in the long run.

Another shortcoming of L-Vox is that computation, altough faster than
the original implementation of the mathemical framework in MATLAB, can
take several hours to finish if the requested voxel size is small
enough. This is explained by the fact that L-Vox was not developped
with the intent of being used operationally, but rather as a testing
ground for the new estimation approach devised by
\cite{pimont2019}. Many optimization approaches can be applied to the
existing codebase to significantly reduce the processing time.

In this article, we will explain how we managed to take the existing
approach employed by L-Vox and make it accessible from both outside
and inside \texttt{Computree} with the same underlying estimation
approach using voxels. This is all in an effort to make L-Vox more
tailored to the operational field of forestry research. This new
version is more interoperable, simpler and faster in terms of
computing time for researchers that have access to consumer grade
hardware.

\bibliography{memoire}
\bibliographystyle{apacite}
\end{document}
